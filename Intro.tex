\section{Introduction}
\indent In the year 1951, Richard Dalitz published a letter~\cite{Dalitz} in which he calculated the rate for the \pizT  decaying into an electron-positron pair (dilepton) and a photon, \pizDal. The calculation assumed that the decay proceeded through a two–photon decay in which one of the photons was virtual and converted internally into an electron-positron pair.  This kind of reaction is now known as a Dalitz decay. The experimental evidence of this decay process was first observed in emulsion plates exposed to the Chicago cyclotron in 1952~\cite{Lord} and a number of experiments performed over the next ten years verified Dalitz’s hypothesis that the \pizDal decay resulted from internal conversion of a virtual photon~\cite{Samios,Lindenfeld,Sargent}. A few years later N. Kroll and W. Wada calculated the framework for Dalitz decays within the QED framework~\cite{KrollWada}, and extended the framework to double Dalitz Decays, in which the \pizT decays into two electron-positron pairs via emission of two virtual photons. Throughout the following years, much work was done to extend the framework of Dalitz decays to heavier mesons, such as \etaT, \omT, \etaTP, and \phiT. With numerous experimental data taken, it was shown that the shape of the dilepton mass spectrum deviated from the QED predictions. Such deviations are attributed to the meson not being point-like, as calculated in QED, but instead to the internal structure of the meson. The virtual photon, that decayed into a dilepton pair, has the ability to probe the structure of meson because, like its on-shell counterpart, emission of a  virtual photon is radiation, which decouples from any strong interaction within the meson when the meson transitions into its decay. Therefore, the information of the transition is encoded into the virtual photon, known as the Transition Form Factor (TFF), and can be characterized as $\left| F(q^2)\right|$, where $q^2$ is the square of the invariant mass of the lepton pair.  The transition form factor can be determined by comparing QED predictions to the experimentally measured rate.
\\
%\indent Currently there are a handful of models which attempt to describe the process within the meson, such models include, but not limited to, Vector Meson Dominance (VMD), Dispersion theory, Chiral Perturbation Theory. In this manuscript, the VMD model will be used, however there is validity in all the models. In VMD, the 
\indent In this proposal we present an experiment to study two channels of which decay via Dalitz decays, \etaDal \ and \phiDal. The \etaT \ and \phiT \ are produced via electro-production, $ep \rightarrow ep\etaP$ and $ep \rightarrow ep\phi$ in Hall B, using the CLAS12 detector. The superior $e^+e^-$/$\pi^+\pi^-$ discrimination of the CLAS12 detector will give access to measurements for which  $e^+e^-$/$\pi^+\pi^-$ branching ratios of $10^{12}$ is achievable.
This proposal is organized as follows. In Section~\ref{sec:motivation} we summarize the current knowledge of Dalitz decays and transition form factors, challenges in dilepton signal quality and how the CLAS12 detector can surpass the current challenges in measuring a TFF of low statistical error. In Section~\ref{sec:kinematics}, an explanation of the kinematics of the decay processes will be given as well as kinematics of main competing backgrounds. In Section~\ref{sec:measurement} a brief discussion on past CLAS analysis will be given, along with a description of analysis techniques that have been used and will be used in a CLAS12 measurement. Also in Section~\ref{sec:measurement}, an explanation of the Monte-Carlo simulations that were performed to extract the acceptances will be given as well as a calculation of expected yield and a validity check on the expected yield from previous CLAS analyses. In Section~\ref{sec:summary} we present the beam time request and a summary of the experiment.



