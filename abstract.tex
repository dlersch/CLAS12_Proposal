\documentclass{aip-cp}

\usepackage[numbers]{natbib}
\usepackage{rotating}
\usepackage{graphicx}
\usepackage{caption}
%%%% environment to highlight text that has changed in version 2.0
\newenvironment{v2}{\color{BrickRed}}{\ignorespacesafterend}

\newlength{\figwidth}
\setlength{\figwidth}{0.9\columnwidth}

\newlength{\qfigheight}
\setlength{\qfigheight}{0.25\textheight}

\newlength{\hfigheight}
\setlength{\hfigheight}{0.5\textheight}

\newcommand{\abbr}[1]{\textsc{\texttt{#1}}}
\newcommand{\desg}[1]{\textit{#1}}
\newcommand{\prog}[1]{\texttt{#1}}

\newcommand{\system}[1]{\abbr{#1}}
\newcommand{\bank}[1]{\abbr{#1}}

\newcommand{\todo}[1]{\textbf{\textcolor{Orange}{#1}}}

\def\Lqcd{\mathcal{L}_{\mathtt{QCD}}}
\def\qfield{\psi}
\def\qbarfield{\overline{\psi}}

\def\th{\textsuperscript{th}}
\def\ith{i\th}

\def\um{{\textmu}m}
\def\d{\mathrm{d}}

%\def\coloronline{(Color online.)\ }
\def\coloronline{}



\newcommand{\particle}[2]{\text{#1\ifthenelse{\equal{#2}{}}{}{$^{#2}$}}}

\newcommand{\e}[1][]{\particle{e}{#1}}
\newcommand{\π}[1][]{\particle{π}{#1}}
\newcommand{\K}[1][]{\particle{K}{#1}}



%%% quarks
\def\uquark{\mathbf{u}}
\def\dquark{\mathbf{d}}
\def\squark{\mathbf{s}}


%%% particles
\def\photon{\text{γ}}
\def\electron{\mathrm{e}^-}
\def\positron{\mathrm{e}^+}
\def\nucleon{\mathrm{N}}
\def\proton{\mathrm{p}}
\def\neutron{\mathrm{n}}
\def\pion{\text{π}}
\def\kaon{\mathrm{K}}
\def\hyperon{\text{Y}}
\def\etameson{\text{η}}
\def\omegameson{\text{ω}}
\def\phimeson{\text{φ}}
\def\rhomeson{\text{ρ}}

\def\piplus{\pion^+}
\def\piminus{\pion^-}
\def\Kplus{\kaon^+}
\def\Kminus{\kaon^-}
\def\Kshort{\kaon^0}
\def\pip{$\pion^+$ }
\def\piz{$\pion^0$ }
\def\pim{$\pion^-$ }
\def\pippim{$\pion^+\pion^-$ }
\def\Ks{$\kaon_s$ }
%%% systems
\def\rf{\mathtt{RF}}
\def\tg{\mathtt{TAG}}
\def\tgrf{\tg_{\rf}}
\def\tof{\mathtt{TOF}}
\def\st{\mathtt{ST}}
\def\beam{\mathrm{beam}}
\def\adc{\mathtt{ADC}}
\def\tdc{\mathtt{TDC}}

%%% descriptives
\def\prop{\mathrm{prop}}
\def\trigoffset{\mathrm{trigger-offset}}
\def\pid{\mathtt{PID}}
\def\vtx{\mathrm{vtx}}

%%% dE/dx
\def\dEdx{\frac{\mathrm{d}E}{\mathrm{d}x}}
\def\dEdxvar{\mathrm{d}E/\mathrm{d}x}

%%% Beta
\def\betasttof{\beta_{\st - \tof}}
\def\betatof{\beta_\tof}
\def\betapid{\beta_\pid}
\def\gammatof{\gamma_\tof}
\def\gammapid{\gamma_\pid}

%%% TAGGER and RF related times
\def\trf{t_\rf}
\def\ttg{t_\tg}
\def\ttgrf{t_{\tg,\rf}}
\def\tpho{t_\mathrm{photon}}
\def\tprop{t_\prop}
\def\ttrigoffset{t_\trigoffset}

\def\tphotgrf{\tpho^{\tg,\rf}}
\def\tphotofpid{\tpho\left(\tof\right)}
\def\tphostpid{\tpho\left(\st\right)}

%%% delta tpho
\def\dtpho{\Delta\tpho}
\def\dtphotofpid{\dtpho\left(\tof-\tg\right)}
\def\dtphosttof{\dtpho\left(\tof-\st\right)}
\def\dtphostpid{\dtpho\left(\st-\tg\right)}

%% particle specific tpho (TOF)
\def\tphotofpidkpf{\tpho\left(\tof,\KplusFast\right)}
\def\tphotofpidkps{\tpho\left(\tof,\KplusSlow\right)}
\def\dtphotofpidkpf{\dtpho\left(\tof-\tg,\KplusFast\right)}
\def\dtphotofpidkps{\dtpho\left(\tof-\tg,\KplusSlow\right)}
\def\dtphotofpidkpkp{\dtpho\left(\tof,\KplusFast-\KplusSlow\right)}

%% particle specific tpho (ST)
\def\dtphostpidkpf{\dtpho\left(\tof-\tg,\KplusFast\right)}
\def\dtphostpidkps{\dtpho\left(\tof-\tg,\KplusSlow\right)}

%%% BEAM energy
\def\Epid{E_\pid}
\def\Ebeam{E_\beam}

%%% TOF Energy deposit
\def\Edeptof{\dEdx\left(\tof\right)}
\def\Edeptofvar{\dEdxvar\left(\tof\right)}


%%% path lengths
\def\lst{\ell_\st}
\def\ltof{\ell_\tof}
\def\lsttof{\ell_{\st-\tof}}

%%% raw subsystem times
\def\tst{t_\st}
\def\ttof{t_\tof}
\def\tsttof{\tsttof}

%%% vertex times
\def\tv{t_\vtx}
\def\tvtgrf{\tv^{\tg,\rf}}
\def\tvtofpid{\tv^\tof}
\def\tvstpid{\tv^\st}

%%% delta vertex times
\def\dtvst{\Delta t_\mathrm{vtx}(\mathtt{TOF-ST})}
\def\dtvpid{\Delta t_\mathrm{vtx}(\mathtt{TOF-PID})}

%%% z position
\def\zv{z_\mathrm{vtx}}
\def\ztgt{z_\mathrm{tgt}}

%%% mass
\def\mtof{m_\tof}
\def\mbook{m_\mathrm{book}}

\def\adcst{\mathtt{ADC}_{\mathtt{ST}}}

\def\M{\mathrm{M}}
\def\MM{\mathrm{MM}}

\def\mmkk{\MM\left(\Kplus\Kplus\right)}

\newcommand{\bra}[1]{\left<#1\right|}
\newcommand{\ket}[1]{\left|#1\right>}
\newcommand{\braket}[2]{\left<#1\middle|#2\right>}

\newcommand*\midhrulefill{%
    \leavevmode\leaders\hrule depth-2pt height 2.4pt\hfill\kern0pt
}

% Document starts
\begin{document}

% Title portion
%\title{Transition Form Factors of Mesons and Baryons with CLAS12}
\title{Transition Form Factors of the $\eta^{\prime}$ Meson and $\phi$ Meson with CLAS12}
\author[aff1]{Michael C. Kunkel\corref{cor1}}
\affil[aff1]{Forschungszentrum J\"ulich, J\"ulich (Germany)}
\corresp[cor1]{m.kunkel@fz-juelich.de}
\author{\textit{For the CLAS Collaboration}}
\maketitle

\begin{abstract}
Transition form factors characterize modifications of the point-like photon-meson vertex due to the structure of the meson. The virtual photon can interact with quarks, therefore it can be used as a probe for the internal structure of mesons and its electromagnetic interaction is calculable with the Kroll-Wada formula, Equation~\ref{eq:kroll};
\begin{equation}
\frac{d\Gamma_{M{\rightarrow l^{+}l^{-}X}}}{dq^{2} d\Gamma_{M{\rightarrow X\gamma}}} = \frac{\alpha}{3\pi q^{2}}\left(\left(1+\frac{q^{2}}{m^{2}_{M}-m^{2}_{X}}\right)^2 - \frac{4m^{2}_{M}q^2}{(m^{2}_{M}-m^{2}_{X})^2}\right)^\frac{3}{2}\left(1-\frac{4m_{l}^{2}}{q^{2}}\right)^{1/2}\left(1+\frac{2m_{l}^{2}}{q^{2}}\right) \bigg|_{\mathrm{Q.E.D}}  \label{eq:kroll} \ . \\
\end{equation}
 where $M$ is the species of meson i.e. $\omega$, $\eta^{\prime}$, $\phi$, etc., $X$ is the child particle in the decay, $m_M$ the mass of the meson, $m_X$ the mass of the child particle, $m_l$ the mass of the lepton species in the decay, i.e. $e^{\pm}$ or $\mu^{\pm}$ and $q$ being the momentum transfer which is identical to the invariant mass of the dilepton. Deviations of Equation~\ref{eq:kroll} represent the internal structure of the meson for pseudoscalar mesons, while for vector mesons the deviation represents the transition from $M \to X$. These deviations are the transition form factor $\left| F(q^2)\right|$ and can be determined by comparing Equation~\ref{eq:kroll} to what is measured experimentally.
 \begin{equation}
 \frac{d\Gamma_{M{\rightarrow l^{+}l^{-}X}}}{dq^{2} d\Gamma_{M{\rightarrow X\gamma}}}\bigg|_{\mathrm{measured}} =   \frac{d\Gamma_{M{\rightarrow l^{+}l^{-}X}}}{dq^{2} d\Gamma_{M{\rightarrow X\gamma}}} \bigg|_{\mathrm{Q.E.D}} \left| F(q^2)\right|^2 \label{eq:kroll_ff} \ . \\
 \end{equation}
 Depending on the decay width of the meson of interest, the transition can be modeled as a simple pole, Equation~\ref{eq:pole}, a complex pole, Equation~\ref{eq:cpole}, or some other function that describes the transition.
 \begin{eqnarray}
\left| F(q^2)\right| = \frac{1}{1-\frac{q^2}{\Lambda^2}} \label{eq:pole} \\
\left| F(q^2)\right|^2 = \frac{\Lambda^2(\Lambda^2 + \gamma^2)}{(\Lambda^2 - q^2)\Lambda^2 \gamma^2} \label{eq:cpole} \ ,
 \end{eqnarray} 
 where $\Lambda$ and $\gamma$ is the mass and width of the virtual vector meson mass, respectively.
 Recent measurements of the transition form factor for $\omega \to \mu^+\mu^- \gamma$ have shown unexpected discrepancies with the Vector Dominance Model and recent models of chiral Lagrangian field theory and dispersion theory attempt to predict the contributions of the virtual vector meson.
Also, the knowledge of the $\eta$ form factor is needed for the interpretation of the $(g_{\mu}-2)/2$ uncertainty. Furthermore, the ratio of the $\eta$ and $\eta^{\prime}$ transition form factor provides information of the mixing angle of the combination of the singlet, $\eta_0$, and nonet, $\eta_8$, which are the components of the physical eigenstates, the $\eta$ and $\eta^{\prime}$ meson.
%\newpage
\\ \\
The CLAS12 detector will be used to identify and measure the $e^+e^-$ decay products by means of the High Threshold Cherenkov Counter (HTCC), Pre-Calorimeter (PCAL) and Electromagnetic Calorimeter (EC). The combination of HTCC+PCAL+EC can provide a rejection factor for single $e^\pm/\pi^\pm$ of up to $10^6$ for momenta less than 4.9~GeV/c with $\approx$ 100\% efficiency. For dileptons ($e^+e^-$ pairs), this rejection factor will be $\approx 10^{12}$, which enables semi-leptonic studies for branching ratios $\approx 10^{-9}$. Precise determination of momenta and angles of the $e^+e^-$ decay products  are the key features available to CLAS12.
A beam time of 100 days at full luminosity of ∼$10^{35}cm^{−2}s^{−1}$ will be sufficient to accumulate enough statistics to perform a complete analysis for transition decays. 

%This will give greater insight into the structure of the $ \eta^{\prime} $ meson than previously measured. 
\end{abstract}


\end{document}
