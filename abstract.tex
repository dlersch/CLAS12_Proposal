\documentclass{aip-cp}

\usepackage[numbers]{natbib}
\usepackage{rotating}
\usepackage{graphicx}
\usepackage{caption}
\def\etaP{\eta^{\prime}}
%%%% environment to highlight text that has changed in version 2.0
\newenvironment{v2}{\color{BrickRed}}{\ignorespacesafterend}

\newlength{\figwidth}
\setlength{\figwidth}{0.9\columnwidth}

\newlength{\qfigheight}
\setlength{\qfigheight}{0.25\textheight}

\newlength{\hfigheight}
\setlength{\hfigheight}{0.5\textheight}

\newcommand{\abbr}[1]{\textsc{\texttt{#1}}}
\newcommand{\desg}[1]{\textit{#1}}
\newcommand{\prog}[1]{\texttt{#1}}

\newcommand{\system}[1]{\abbr{#1}}
\newcommand{\bank}[1]{\abbr{#1}}

\newcommand{\todo}[1]{\textbf{\textcolor{Orange}{#1}}}

\def\Lqcd{\mathcal{L}_{\mathtt{QCD}}}
\def\qfield{\psi}
\def\qbarfield{\overline{\psi}}

\def\th{\textsuperscript{th}}
\def\ith{i\th}

\def\um{{\textmu}m}
\def\d{\mathrm{d}}

%\def\coloronline{(Color online.)\ }
\def\coloronline{}



\newcommand{\particle}[2]{\text{#1\ifthenelse{\equal{#2}{}}{}{$^{#2}$}}}

\newcommand{\e}[1][]{\particle{e}{#1}}
\newcommand{\π}[1][]{\particle{π}{#1}}
\newcommand{\K}[1][]{\particle{K}{#1}}



%%% quarks
\def\uquark{\mathbf{u}}
\def\dquark{\mathbf{d}}
\def\squark{\mathbf{s}}


%%% particles
\def\photon{\text{γ}}
\def\electron{\mathrm{e}^-}
\def\positron{\mathrm{e}^+}
\def\nucleon{\mathrm{N}}
\def\proton{\mathrm{p}}
\def\neutron{\mathrm{n}}
\def\pion{\text{π}}
\def\kaon{\mathrm{K}}
\def\hyperon{\text{Y}}
\def\etameson{\text{η}}
\def\omegameson{\text{ω}}
\def\phimeson{\text{φ}}
\def\rhomeson{\text{ρ}}

\def\piplus{\pion^+}
\def\piminus{\pion^-}
\def\Kplus{\kaon^+}
\def\Kminus{\kaon^-}
\def\Kshort{\kaon^0}
\def\pip{$\pion^+$ }
\def\piz{$\pion^0$ }
\def\pim{$\pion^-$ }
\def\pippim{$\pion^+\pion^-$ }
\def\Ks{$\kaon_s$ }
%%% systems
\def\rf{\mathtt{RF}}
\def\tg{\mathtt{TAG}}
\def\tgrf{\tg_{\rf}}
\def\tof{\mathtt{TOF}}
\def\st{\mathtt{ST}}
\def\beam{\mathrm{beam}}
\def\adc{\mathtt{ADC}}
\def\tdc{\mathtt{TDC}}

%%% descriptives
\def\prop{\mathrm{prop}}
\def\trigoffset{\mathrm{trigger-offset}}
\def\pid{\mathtt{PID}}
\def\vtx{\mathrm{vtx}}

%%% dE/dx
\def\dEdx{\frac{\mathrm{d}E}{\mathrm{d}x}}
\def\dEdxvar{\mathrm{d}E/\mathrm{d}x}

%%% Beta
\def\betasttof{\beta_{\st - \tof}}
\def\betatof{\beta_\tof}
\def\betapid{\beta_\pid}
\def\gammatof{\gamma_\tof}
\def\gammapid{\gamma_\pid}

%%% TAGGER and RF related times
\def\trf{t_\rf}
\def\ttg{t_\tg}
\def\ttgrf{t_{\tg,\rf}}
\def\tpho{t_\mathrm{photon}}
\def\tprop{t_\prop}
\def\ttrigoffset{t_\trigoffset}

\def\tphotgrf{\tpho^{\tg,\rf}}
\def\tphotofpid{\tpho\left(\tof\right)}
\def\tphostpid{\tpho\left(\st\right)}

%%% delta tpho
\def\dtpho{\Delta\tpho}
\def\dtphotofpid{\dtpho\left(\tof-\tg\right)}
\def\dtphosttof{\dtpho\left(\tof-\st\right)}
\def\dtphostpid{\dtpho\left(\st-\tg\right)}

%% particle specific tpho (TOF)
\def\tphotofpidkpf{\tpho\left(\tof,\KplusFast\right)}
\def\tphotofpidkps{\tpho\left(\tof,\KplusSlow\right)}
\def\dtphotofpidkpf{\dtpho\left(\tof-\tg,\KplusFast\right)}
\def\dtphotofpidkps{\dtpho\left(\tof-\tg,\KplusSlow\right)}
\def\dtphotofpidkpkp{\dtpho\left(\tof,\KplusFast-\KplusSlow\right)}

%% particle specific tpho (ST)
\def\dtphostpidkpf{\dtpho\left(\tof-\tg,\KplusFast\right)}
\def\dtphostpidkps{\dtpho\left(\tof-\tg,\KplusSlow\right)}

%%% BEAM energy
\def\Epid{E_\pid}
\def\Ebeam{E_\beam}

%%% TOF Energy deposit
\def\Edeptof{\dEdx\left(\tof\right)}
\def\Edeptofvar{\dEdxvar\left(\tof\right)}


%%% path lengths
\def\lst{\ell_\st}
\def\ltof{\ell_\tof}
\def\lsttof{\ell_{\st-\tof}}

%%% raw subsystem times
\def\tst{t_\st}
\def\ttof{t_\tof}
\def\tsttof{\tsttof}

%%% vertex times
\def\tv{t_\vtx}
\def\tvtgrf{\tv^{\tg,\rf}}
\def\tvtofpid{\tv^\tof}
\def\tvstpid{\tv^\st}

%%% delta vertex times
\def\dtvst{\Delta t_\mathrm{vtx}(\mathtt{TOF-ST})}
\def\dtvpid{\Delta t_\mathrm{vtx}(\mathtt{TOF-PID})}

%%% z position
\def\zv{z_\mathrm{vtx}}
\def\ztgt{z_\mathrm{tgt}}

%%% mass
\def\mtof{m_\tof}
\def\mbook{m_\mathrm{book}}

\def\adcst{\mathtt{ADC}_{\mathtt{ST}}}

\def\M{\mathrm{M}}
\def\MM{\mathrm{MM}}

\def\mmkk{\MM\left(\Kplus\Kplus\right)}

\newcommand{\bra}[1]{\left<#1\right|}
\newcommand{\ket}[1]{\left|#1\right>}
\newcommand{\braket}[2]{\left<#1\middle|#2\right>}

\newcommand*\midhrulefill{%
    \leavevmode\leaders\hrule depth-2pt height 2.4pt\hfill\kern0pt
}

% Document starts
\begin{document}

% Title portion
%\title{Transition Form Factors of Mesons and Baryons with CLAS12}
\title{Transition Form Factors of the $\eta^{\prime}$ and $\phi$ Mesons with CLAS12}
\author[aff1]{M. C. Kunkel\corref{cor1}}
\author[aff2]{M. J. Amaryan}
\author[aff1]{D. Lersch}
\author[aff1]{J. Ritman}
\author[aff1]{S. Schadmand}
\author[aff1]{X. Song}
\affil[aff1]{Forschungszentrum J\"ulich, J\"ulich (Germany)}
\affil[aff2]{Old Dominion University (U.S.A.)}
\corresp[cor1]{m.kunkel@fz-juelich.de}
\author{\textit{for the CLAS Collaboration}}
\maketitle

\begin{abstract}
Dalitz decays are radiative decays in which the photon is virtual and subsequently produces an electron positron pair, $P\rightarrow l^+l^-X$. Such decays serve as an important tool used to reveal the internal structure of hadrons and the interaction mechanisms between photons and hadrons. Furthermore, assuming point-like particles, the electromagnetic interaction is calculable within QED by the Kroll-Wada formula. Transition form factors quantify deviations from the QED decay rate. They characterize modifications of the point-like photon-meson vertex due to the structure of the meson. For the $\etaP$ meson this deviation represents the internal structure of the meson, while for the $\phi$ meson the deviation represents the transition from $\phi \to \eta$. The transition form factor can be characterized as $\left| F(q^2)\right|$, where $q^2$ is the square of the invariant mass of the lepton pair, and can be determined by comparing QED predictions to the experimentally measured rate.
 \\ 
 \indent Measurements with the highest scientific impact on the determination of the transition form factor have been performed in the space-like region ($\mathrm{q}^2<0$) in collider experiments. However, due to experimental limitations (e.g. $\pi^{\pm}$ contamination in lepton sample, low branching fractions, external conversion contamination), transition form factors in the time-like region ($\mathrm{q}^2>0$) have not yet been precisely determined. Recent measurements of the time-like transition form factor for $\etaP \to e^+e^- \gamma$ have been performed by the BESIII collaboration with insufficient statistical precision, therefore the proper theoretical description cannot be determined. 
\\
\indent From previous CLAS analyses using the g12 data set, it was preliminarily shown that measurements of the time-like transition form factor were achievable, but without the statistical precision needed to be competitive. Therefore, we propose to use CLAS12 to focus on the dilepton decay channels from the reactions $ep\rightarrow e^{\prime}p\etaP$ and $ep\rightarrow e^{\prime}p\phi$, where $\etaP \to e^+e^- \gamma$ and $\phi \rightarrow \eta e^+e^-$. The CLAS12 detector will be used to identify and measure the $e^+e^-$ decay products by means of the High Threshold Cherenkov Counter (HTCC), Pre-Calorimeter (PCAL) and Electromagnetic Calorimeter (EC). The combination of HTCC+PCAL+EC can provide a rejection factor for single $e^\pm/\pi^\pm$ of up to $10^6$ for momenta less than 4.9~GeV/c with $\approx$ 100\% efficiency. For dileptons ($e^+e^-$ pairs), this rejection factor will be $\approx 10^{12}$, which enables dilepton studies for branching ratios $\approx 10^{-9}$. Precise determination of momenta and angles of the $e^+e^-$ decay products are the key features available to CLAS12. The momentum and angle of final state photons will be determined in CLAS12
by using the PCAL and EC. Consequently, the photon in the process $\eta^{\prime} \rightarrow e^+e^- \gamma$ and the photons in the process $\phi \rightarrow e^+e^- \eta \rightarrow e^+e^- \gamma \gamma$ will be detected. Preliminary studies using the CLAS12 simulation suite have shown that a beam time of 100 days, at full luminosity, will accumulate a data sample at least one order of magnitude larger in statistics than the most current $\etaP \to e^+e^- \gamma$ and $\phi \rightarrow \eta e^+e^-$ measurement. 
\\
\\ Conditions to run with RunGroup A:
\begin{itemize}
	\item Standard CLAS12 setup
	\item Liquid hydrogen target
	\item 75\% torus field
	\item HTCC+PCAL+EC trigger
\end{itemize}
\indent There will be no adverse effects from running the proposed beamtime in parallel with RunGroup A.
%of ∼$10^{35}cm^{−2}s^{−1}$ 
%This will give greater insight into the structure of the $ \eta^{\prime} $ meson than previously measured. 
\end{abstract}

\end{document}
