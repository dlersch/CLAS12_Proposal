\subsubsection{$\etaP \to \gamma \gamma$  Decay}\label{sec:piz.gg}
As shown in Fig.~\ref{fig:piz.gamgam}, the two photon decay can be expressed in terms of the respective momentum, $P_p$($\eta^{\prime}$)$\to \gamma(\epsilon_1,p) \gamma(\epsilon_2,k)$, where $\epsilon_1$ and $\epsilon_2$ are the polarizations of the photons with 4-momenta $p$ and $k$. Dropping the nomenclature ($\eta^{\prime}$) in $P_p$($\eta^{\prime}$), the four momentum of the decaying meson is $P_p= p+k$. Using the Feynman rules as given in~\cite{peskin}and~\cite{halzen}, which are Lorentz and gauge invariant and also parity conserving, the amplitude can be solved to be:

\begin{align}\label{eq:piz.gg.amp}
 {\cal M}(P_P \to \gamma(\epsilon_1,p) \gamma(\epsilon_2,k))= {M}_P(p^2=0,k^2=0) \varepsilon_{\mu\nu\rho\sigma}\epsilon_1^\mu p^\nu \epsilon_2^\rho k^\sigma
\end{align}
where $\varepsilon_{\mu\nu\rho\sigma}$ is the antisymmetric metric tensor. The form factor, ${M}_P(p^2=0,k^2=0)$, contains information of the decaying meson and since the decay products are on-shell photons, which are massless, ${M}_P$ is a constant given as;
\begin{align}\label{eq:decay.constants}
 {M}_P=\begin{cases}
         {\displaystyle\frac{\alpha}{\pi f_{\pi}}} & \mbox{if $P=\etaP$};\\
        {\displaystyle\frac{\alpha}{\pi f_\pi} \frac{1}{\sqrt{3}} }\left( \frac{f_\pi}{f_8} \cos\theta_{mix} -2\sqrt{2} \frac{f_\pi}{f_0} \sin\theta_{mix} \right)& \mbox{if $P=\eta$};\\
        {\displaystyle\frac{\alpha}{\pi f_\pi} \frac{1}{\sqrt{3}}} \left( \frac{f_\pi}{f_8} \sin\theta_{mix} +2\sqrt{2} \frac{f_\pi}{f_0} \cos\theta_{mix} \right)& \mbox{if $P=\eta'$} \,
\end{cases}
\end{align}
where $\alpha=e^2/4\pi \approx 1/137$ is the fine structure constant, $f_\pi \approx 92.4 \,{\rm MeV}$ is the physical value of the pion-decay constant and $f_0 \approx 1.04 f_\pi$ and $f_8 \approx 1.3 f_\pi$ are the singlet and octet Pseudo-Goldstone meson decay constants.

\subsubsection{\emph{Squared Matrix Element}}\label{subsec:etaPMartix}
The squared matrix element of the decay $P_P \to \gamma(\epsilon_1,p) \gamma(\epsilon_2,k)$ is given by
\begin{align}\label{eq:piz.gg}
\left|{\cal  M}(P_{P}\rightarrow\gamma(\epsilon_{1},p)\gamma(\epsilon_{2},k))\right|^{2}=\left|M_{P}\right|^{2}\varepsilon_{\mu\nu\rho\sigma}\varepsilon_{\mu^{\prime}\nu^{\prime}\rho^{\prime}\sigma^{\prime}}\epsilon_{1}^{\mu}p^{\nu}\epsilon_{2}^{\rho}k^{\sigma}\epsilon_{1}^{\mu^{\prime}}p^{\nu^{\prime}}\epsilon_{2}^{\rho^{\prime}}k^{\sigma^{\prime}}
%
\end{align}
which can be simplified to;
\begin{align}\label{eq:piz.gg.simplify}
\left|{\cal M}(P_{P}\to\gamma(p)\gamma(k))\right|^{2}=\left|M_{P}\right|^{2}\varepsilon_{\mu\nu\rho\sigma}\varepsilon^{\mu\nu}_{\quad \rho^{\prime}\sigma^{\prime}}p^{\rho}p^{\rho^{\prime}}k^{\sigma}k^{\sigma^{\prime}}
%
\end{align}
by assuming that the polarizations of the photons remain unobserved, as they are in \abbr{CLAS}. Therefore the photon polarization vectors can be summed using Eq.~5.75 from~\cite{peskin} which reads as;
\begin{align}
\sum\limits_{polarizations} \epsilon_{\mu} \epsilon_{\mu^{\prime}} \to -g_{\mu\mu^{\prime}} 
\end{align}
As indicated in ~\cite{peskin}, the right arrow indicates that this is not an actual equality, but the solution is valid as long as both sides are dotted into Eq.~\ref{eq:piz.gg}. The antisymmetric tensor, $\varepsilon_{\mu\nu\rho\sigma}\varepsilon^{\mu\nu}_{\quad \rho^{\prime}\sigma^{\prime}}$ is simplified using  Eq.~A.30 of \cite{peskin}; 
\begin{align}\label{eg:antiT_ID}
\varepsilon_{\mu\nu\rho\sigma}\varepsilon^{\mu\nu}_{\quad \rho^{\prime}\sigma^{\prime}} = -2(g_{\rho\rho^{\prime}}g_{\sigma\sigma^{\prime}} - g_{\rho\sigma^{\prime}}g_{\rho^{\prime}\sigma})\\
\end{align}
Applying Eq.~\ref{eg:antiT_ID} to Eq.~\ref{eq:piz.gg.simplify} results in;
\begin{align}\label{eq:piz.gg.reduced}
\left|{\cal M}(P_{P}\to\gamma(p)\gamma(k))\right|^{2}=\left|M_{P}\right|^{2}(-2)(p^2k^2 - (p\cdot k)^2) \ .
%
\end{align}
Substituting
\begin{align}
(p + k)^2 = p^2 + k^2 +2 (p\cdot k) \ ,
\end{align}
and applying $p^2= k^2=0$, since both photons are massless because they are on-shell, we can derive the final expression of the squared amplitude of the decay $P_P \to \gamma(\epsilon_1,p) \gamma(\epsilon_2,k)$ as;
\begin{align}\label{eq:piz_gg_amp_final}
\left|{\cal M}(P_{P}\to\gamma(p)\gamma(k))\right|^{2}= \left|M_{P}\right|^{2}\frac{1}{2}(p+k)^{4} = \frac{1}{2}\left|M_{P}\right|^{2}m_{P}^{4}
\end{align}
where $m_P^4$ is the mass of the \etaTP derived from the 4-momenta conservation equation $(p+k)^4 = m_P^4$
\subsubsection{\emph{Decay rate}}
The decay rate of a two-body decay is explained in Equation 46.17 of~\cite{pdg2014} as
\begin{align}\label{eq:pdg.2body}
d\Gamma = \frac{1}{32 \pi^2} A \left|{\cal M}\right|^2\frac{\left|\bf{p_1}\right|}{m_p^2}d\Omega \ ,
\end{align}
where $d\Omega$ is the solid angle of particle 1 and $A$ is the symmetry factor which appears because of the Bose symmetry of the two
outgoing photons. Substituting the square matrix element from Eq.~\ref{eq:piz_gg_amp_final} into Eq.~\ref{eq:pdg.2body} and integrating over the solid angle yields;
\begin{align}
\Gamma_{P\rightarrow\gamma\gamma} = \frac{1}{32\pi^{2}} \frac{1}{2} \left|{\cal M}(P_{P}\to\gamma(p)\gamma(k))\right|^{2} \frac{\left|\bf{p}\right|}{m_{P}^2} 4 \pi = \frac{1}{32 \pi}\left|M_{P}\right|^{2}m_{P}^{2}\left|\bf{p}\right|
\end{align} 
Finally, in the center-of-mass (C.M.) frame of the decaying meson, $\bf{p} = E_{\gamma}^{C.M.} = \frac{m_p}{2}$, we find the final expression of the decay rate of $P_P \to \gamma(\epsilon_1,p) \gamma(\epsilon_2,k)$ as;
\begin{align}\label{eq:piz.gg.decay.final}
\Gamma_{P\rightarrow\gamma\gamma} = \frac{1}{64\pi} \left|M_{P}\right|^{2}m_{P}^{3} \ .
\end{align}
\subsection{$\phi \to \eta \gamma$ Decay}\label{sec:phi.etag}
The $\phi \rightarrow \eta \gamma $ decay is analogous to the $\etaP \rightarrow \gamma \gamma$ decay by replacing the initial pseudoscalar meson $P_p$with a vector meson $V_p$ and one of the $\gamma$ propagators with $\eta$. In this substitution 
\begin{align}\label{eq:phi.etag.amp}
{\cal M}(V_P(\epsilon_1) \to \eta(p) \gamma(\epsilon_2,k)) = {M}_V(p^2=m_{\eta},k^2=0) \varepsilon_{\mu\nu\rho\sigma}\epsilon_1^\mu p^\nu \epsilon_2^\rho k^\sigma
\end{align}
again, $\varepsilon_{\mu\nu\rho\sigma}$ is the antisymmetric metric tensor and the form factor, ${M}_V(p^2=\eta,k^2=0)$, contains information of the $\phi-\eta$ transition.

\subsubsection{\emph{Decay rate}}
The algebra for solving the squared matrix element is similar to Sec.~\ref{subsec:etaPMartix}, however since now the initial meson has polarization, a factor of 1/3~\cite{Hanhart2015} is introduced. The decay rate is also similar within a factor of 1/3 and can be represented as:
\begin{align}\label{eq:phi.etag.decay.final}
\Gamma_{V\rightarrow\eta\gamma} = \frac{1}{3} \frac{1}{64\pi} \left|M_{V}\right|^{2}\left(\frac{m_{\phi} - m_{\eta}}{m_{\phi}} \right)^{3} \ .
\end{align}

%

